Each node on the baseboard contains a dedicated communications line. The protocol used by the baseboard to communicate with the blocks uses only the power and ground wires connected to each block and is based on the \emph{Dallas 1-Wire Protocol}.

The communications protocol is a \emph{timing-based} protocol -- each signal is transmitted by holding the power line low for a designated timing period. There are three elementary types of messages conveyed in this manner: \texttt{RESET}, \texttt{1-BIT}, and \texttt{0-BIT}. \texttt{RESET} is responsible for alerting the line that a message is about to be transmitted while \texttt{1-BIT} and \texttt{0-BIT} are used in conjunction to transmit arbitrary commands and responses between entities on the line. Since signal transmission briefly disables the power line, each block is powered during this period by a capacitive reserve power circuit. Additionally, as part of the protocol, each block microcontroller is preprogrammed with a unique ID.

Interactions between a block and the board are typically conducted as follows. The baseboard transmits a \texttt{RESET} command on a node that tells the blocks on that node that an interaction is about to happen. Next, the baseboard sends a command, optionally prefixed with a target address, in the form of a sequence of \texttt{0-BIT} and \texttt{1-BIT} signals. Once the command is executed by the target(s), a result is transmitted to the baseboard in the same way. In the absence of a target address, the command is addressed to \emph{all} blocks on the communications line.