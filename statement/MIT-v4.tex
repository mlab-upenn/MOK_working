\documentclass[11pt]{letter}
\usepackage{graphicx}
\usepackage{wrapfig}
\usepackage{lipsum}
\usepackage{scrextend}
\usepackage{verbatim, hyperref}
\usepackage{todonotes}
\usepackage[tmargin=1in,bmargin=1in,lmargin=1.25in,rmargin=1.25in]{geometry}
%opening
\title{}
\author{}

\begin{document}
\newcommand{\comm}[1]{}

Vincent Pacelli \hfill \hfill \href{mailto:pacelliv@seas.upenn.edu}{pacelliv@seas.upenn.edu}\\
\rule{\textwidth}{0.1pt}

My coursework and independent projects at the University of Pennsylvania School of Engineering and Applied Science have fully equipped me with the substantive knowledge and practical experience to be a strong contributor to the MIT teams conducting research in the fields of cyber-physical and robotic systems, which is my primary area of academic interest.

I have completed significant graduate and undergraduate studies and projects in applied subjects, such as \emph{operating systems}, \emph{solid-state circuits}, and \emph{embedded systems}, as well as material that makes up the theoretical underpinnings of CPS, such as \emph{linear systems, theory of embedded computation, game theory, probability}, and \emph{stochastic processes}.  My curriculum has prepared me to both explore new theoretic concepts and implement experiments to establish their potency.  As a Ph.D. student, I will continue to increase my engineering expertise by acquiring additional mathematical and practical tools in order to solve the unique problems associated with enabling machines to efficiently and safely make the decisions required to complete tasks, even where the environment is dynamic and uncertain.

In addition to coursework, I have participated in internships and independent study. During these experiences, I have repeatedly tackled projects that link theory with application. For example, when given a choice of tasks as an intern at NASA Langley's Safety Critical Avionics Branch, I elected to help develop a real-time UAV mission optimizer because it afforded me the opportunity to address challenging problems through the employment of probabilistic algorithms. The project required overcoming the not-uncommon difficulty of how to autonomously prioritize a series of objectives based on their costs and rewards in response to changing mission parameters. Since finding an optimal solution to this problem is NP-Hard and thus intractable in real-time, my algorithms utilized \emph{Bayesian methods} to construct a high-performing plan by estimating which objectives are most likely to appear next in an optimal path.  While a probabilistic algorithm sacrifices some amount of optimality, it permits meeting strict real‑time requirements and enables quick reaction to unforeseen events.  My work completed during this internship will be published in an upcoming branch white paper.

Similarly, I selected my senior design project in order to apply recent results in controller synthesis to real-world mission planning problems. Over the course of the year, I am working in collaboration with Penn's GRASP Lab to develop a multi-agent UAV system to help aid disaster response teams. Users will be able to persistently scan areas remotely via UAV on fixed intervals while avoiding the errors of erroneously plotting paths that result in vehicle collisions or running out of fuel which are typical of human-orchestrated systems. In contrast to the probabilistic approach presented in the previous application, this system will utilizes recent research which suggests that, by converting timing specifications and points-of-interest input from the user to the rich semantics of \emph{bounded linear temporal logic}, a ``correct-by-construction'' automaton can be generated that represents our vehicles and the environment, which in turn renders it possible to synthesize the necessary vehicle controllers to meet the mission specification without collisions or other safety issues.  As a result, the system will be more reliable and require less human interaction than it would otherwise.

Going forward, I intend to continue to explore problems that relate \emph{planning, verification,} and \emph{control} of autonomous systems.  These three fields are strongly intertwined and their importance lies in their relationship to a characteristic question of autonomy:  what are appropriate and desirable uses of autonomous technology and how may such uses be achieved? Consistent advances in the fields of computer architecture and chip fabrication permit imbuing physical systems with increasingly vast computational power, yet brute-force alone cannot provide either the nuanced behavior or safety assurances demanded of cyber-physical systems intended to augment or replace human workers. To achieve these goals, it is imperative to develop abstractions that better lend themselves to encapsulating both digital computation and physical control of hybrid systems. By considering these subsystems holistically, we can competently and strategically interweave high-level intelligence with low-level dynamics to create safer, more capable, and more efficient systems with higher degrees of autonomy.  

Furthermore, if we are to trust the next generation of autonomous vehicles, medical devices, and other high-performance systems, such systems must be capable of handling unpredictable events gracefully -- either by adapting safely to the new scenario or by failing in a responsible manner. While model-based design provides very powerful tools for developing cyber-physical systems, it typically relies on assumptions that may be violated in extreme, real-world scenarios. Thus, a case is made for developing methods and tools that are either robust or capable of adapting to change. Even with the great strides that have recently been made in areas such as formal verification and different styles of intelligent control, the work that remains to be done continues to grow as the rapidly developing capabilities of autonomous systems are more than matched by the expanding vision of their potential application.  These are the sorts of problems and challenges that are especially intriguing to me and I am excited to be part of an academic and professional community that is charged with addressing them.

MIT is the perfect institution for me to take the next step in my academic career, in particular because of the breadth and depth of MIT’s pioneering research in autonomous cyber-physical systems and their application.  \textbf{Dr. Sertac Karaman's} work in robotics and control theory with application in systems operating in challenging, multi-agent environments; \textbf{Dr. Russ Tedrake's} research in motion planning for dexterous tasks and interacting with uncertain environments; \textbf{Dr. Brian William's} research on the applicability of commonsense reasoning in self-maintaining autonomous explorers; and \textbf{Dr. Nicholas Roy's} research into the integration of spatial and temporal reasoning into computation systems, especially autonomous decision-making based upon incomplete information, are all projects which would offer me the opportunity to apply and expand my knowledge and experience.  For all of these reasons, I would be excited to be part of the MIT, where I may formulate, develop, and solve interesting problems in cyber-physical systems.
%\rule{\textwidth}{1pt}

\begin{comment}
{\bf \emph{References}}

[1] "Cyber-Physical Systems." Baheti and Gill

[2] "Optimization and Control of Cyber-Physical Vehicle Systems." Bradley and Atkins

\end{comment}

\end{document}