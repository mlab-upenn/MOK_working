\documentclass[]{letter}
\usepackage{graphicx}
\usepackage{wrapfig}
\usepackage{lipsum}
\usepackage{scrextend}
\usepackage{verbatim, hyperref}
\usepackage{todonotes}
\usepackage[tmargin=1in,bmargin=1in,lmargin=1.25in,rmargin=1.25in]{geometry}
%opening
\title{}
\author{}

\begin{document}
\newcommand{\comm}[1]{}

Vincent Pacelli \hfill \hfill \href{mailto:pacelliv@seas.upenn.edu}{pacelliv@seas.upenn.edu}\\
\rule{\textwidth}{0.1pt}

%\todo[inline, color=green!40]{I moved this paragraph up because it is more interesting lead in, needs to be updated for new context...}
In recent years, we have witnessed the rapid proliferation of cyber-physical systems. However, as we demand increasingly more intricate behavior from such systems, developers of such systems are asked to provide stronger guarantees about their devices: how can we trust these systems to behave both efficiently and safely? If we are to be able to trust autonomous agents to make decisions about critical (and not so critical) tasks, then the next generation of vehicles, medical devices, sensors, and robotics must be accompanied by \emph{proofs of their own efficacy}. In short, such systems must be engineered, not crafted, and, as a Ph.D. candidate, I intend to develop new methods for creating sophisticated behaviors under strong safety and performance guarantees.

As an undergraduate at the University of Pennsylvania, I have successfully prepared myself to begin conducting independent research in the field of cyber-physical systems as a Ph.D. candidate. I have completed significant coursework and projects in various applied subjects, such as \emph{operating systems}, \emph{solid-state circuits}, and \emph{embedded systems}. Additionally, I have studied many of the subjects that make up the theoretical underpinnings of the field, such as \emph{linear systems}, \emph{theory of embedded computation}, \emph{game theory}, \emph{probability}, and \emph{stochastic processes} and have sought to apply these topics in practical applications whenever possible.

For example, when given a choice of tasks as an intern at NASA Langley's Safety Critical Avionics Branch, I elected to help develop a real-time UAV mission optimization algorithm because it allowed me to apply probability theory to a real-world problem. Determining an order to complete various objectives based on their costs and rewards is an often necessary component of an autonomous system. However, the NP-Hard nature of this problem prevents real-time adaptation to changing mission parameters. Our solution utilizes \emph{Bayesian methods} to construct high-performing approximate solutions out of objectives that are most likely to lead to better performing paths. This method allows us to plan new missions rapidly when mission parameters or constraints change, and my work as an intern will be published in a branch white paper.

Similarly, in collaboration with Penn's GRASP lab, I selected my senior design project -- a multi-agent UAV system for post-disaster evaluation -- because it allows me to apply recent results in controller synthesis. Our system consists of a custom interface for planning and analyzing missions for our thermal imaging UAV. While project resources have limited us to a single physical UAV, we intend to support multiple agents with different sensors. One application of our system is for teams to persistently scan areas remotely via UAV on some fixed interval. A human orchestrating such a task is liable to plot paths that result in collisions or running out of fuel. However, we intend to leverage recent research to attack the problem by converting timing specifications and points of interest to the rich semantics of \emph{bounded linear temporal logic}. Then, we can generate an automaton representing our vehicles and the environment and synthesize the necessary vehicle controllers to meet the mission specification without collisions or other safety issues. As a result, our system will be both more flexible and robust. Experiences such as these make me uniquely qualified to conduct independent research in autonomous cyber-physical or robotic systems. As such, I am most interested in working with professors such as {\bf{Dr. Sertac Karaman, Dr. Russ Tedrake, Dr. Brian Williams,}} or {\bfseries{Dr. Nicholas Roy}}.

Specifically, I am most interested in exploring topics related to \emph{planning, verification}, and \emph{control} of autonomous systems. Advancements in computer architecture and chip fabrication have allowed us to imbue physical systems with vast computational power. However, in order to create the compelling utilities and tools that futurists envision -- particularly those involved in safety-critical applications -- we need to develop better abstractions that ``enable [the] seamless integration of control, communication, and computation must be developed for rapid design'' [1]. Such abstractions are necessary to better answer the two fundamental question all autonomous systems seek answer: what do we need to do and how may it be done safely? These questions emphasize the tight coupling between planning, verification, and control and their intricate relationships are ripe for exploration. By considering these subsystems holistically, we may ``more efficient, safe, secure and capable systems as we increase the level of autonomy in CPS devices and vehicles'' [2]. In order to better reason about, dissect, and attack problems in this space, it is imperative that we develop better mathematical models that dually address both the physical and computational aspects of such systems. Thus, I am eager to further develop and expand my mathematical toolkit so I may more effectively build, describe, and analyze these formalisms. Moreover, it is important in developing these models we do not lose sight of the real-world constraints of the problems we aim to address and verify their validity through experimentation. To this end, I intend to continue to grow my technical abilities to encompass practical skills so that I may demonstrate the potency of my methods on real systems.

%\todo[inline]{You have listed nothing here that is unique, almost everyone apply will have done these things. At least at this level of generality...be more specific say something more personal }


%\todo[inline]{You are leading with the most boring part...}
%My coursework can be roughly divided into two categories: theoretical underpinnings and technical foundations.  On the theoretical front, I have completed or am now taking graduate courses in \emph{linear systems} and \emph{theory of embedded computation}, which provide the groundwork of continuous and discrete state evolution and actuation; \emph{probability} and \emph{stochastic processes}, which provide a foundation for many popular perception and planning algorithms; \emph{game theory}, which provides a means of synthesizing multi-agent controllers; and \emph{control theory}, which bridges the gap between the practical and theoretical aspects of developing such systems.  On the technical side, through significant projects in courses including \emph{embedded systems}, \emph{operating systems}, and \emph{solid-state circuits}, I have developed the practical skills necessary to design and build both the software and hardware components of an embedded system.  As a Ph.D. student, I hope to continue to add to my skill set, particularly by learning new mathematical and practical tools with which I may attack problems.

%\todo[inline]{I think the following paragraph should be split up with more detail, see below}
%Because I believe that the best way to develop understanding is to apply knowledge gained from studies, I have supplemented my classroom education with rigorous independent projects that featured both theoretical and practical components. Prior to my arrival at Penn, I developed a software emulation of the Nintendo Gameboy in an effort to better understand computer architecture. Then, in my second year, I gained practical experience developing research software for characterizing infant jaw motion, some of which is still actively being used by hospitals. More recently, I interned at NASA Langley's Safety Critical Avionics Branch, where I helped develop software for routing UAVs in the presence of changing mission parameters. My work will be published as part of an upcoming branch white paper. In addition, I am actively involved as an embedded systems engineer in Penn’s xLab, where I help produce resource-constrained embedded systems for context-aware toys.  Ensuring a robust user experience entails meeting strict performance and size requirements and which requires creativity in design and skill in execution. Finally, in collaboration with Penn's GRASP Lab, I am currently working on my senior design project -- a multi-agent UAV system to help disaster response teams conduct searches. We intend to incorporate recent results from multi-agent planning research, such as mission synthesis from logic specifications, to improve the efficiency, robustness, and usability of the system. In essence, this project epitomizes the major theme of my engineering education to date: by understanding \emph{theoretic abstractions}, we can design better \emph{physical systems}.


%More specifically, I intend to explore problems that relate the planning and control of autonomous systems.  While Moore's Law provides us with smaller and more efficient computing platforms, brute force alone cannot achieve the sophisticated interactions promised by futurists like Asimov and Roddenberry and which are worthy of our aspiration.  Instead, we need to continue to develop intelligent algorithms to solve problems that humans do naturally, such as where to go and how to get there.  Recent techniques, such as “correct-by-construction” controller synthesis and trajectory generation via LQR-Trees, have greatly improved the tractability of autonomous systems but significant further research is still required -- especially in scenarios involving dangerous obstacles and non-linear or stochastic systems. 
%\todo[inline]{This section could be more specific, dropping jargon without showing you understand it is not great}
%\todo[inline]{Perhaps this is where you want to explain your GRASP project (the components, techniques, and objectives) in more detail}
 %Due to the intricacy of these problems and the importance of finding solutions to them in our increasingly augmented world, I relish the opportunity to address them, and, in this regard, I would be very interested in working with {\bf{Dr. Sertac Karaman, Dr. Russ Tedrake, Dr. Brian Williams,}} or {\bfseries{Dr. Nicholas Roy}}.
%\todo[inline]{I bolded the names, this is generally recommended so people don't have to search who to forward these on to}

In summary, my undergraduate education and work experience in the theoretical and practical components of cyber‑physical systems has prepared me to conduct research in autonomous systems. The extracurricular projects I have completed display my significant technical skills. In my experience, the most difficult problems in cyber- physical systems are often ones that humans meet with little effort: moving, planning, coordinating, etc. I see the relationship between planning, verification, and control as a space replete with important and fascinating problems. I hope to have the opportunity to be part of the MIT team that is developing unique solutions to these problems and enabling a more intelligent, safe, and efficient world.
%\rule{\textwidth}{1pt}

{\bf \emph{References}}\\

[1] "Cyber-Physical Systems." Baheti and Gill

[2] "Optimization and Control of Cyber-Physical Vehicle Systems." Bradley and Atkins
\end{document}