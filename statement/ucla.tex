\documentclass[]{article}
\usepackage{graphicx}
\usepackage{wrapfig}
\usepackage{lipsum}
\usepackage{scrextend}
\usepackage{verbatim}
%opening
\title{}
\author{}

\begin{document}


\begin{addmargin}[0em]{2em}
\verb!> Hello, World! !
\vspace{2mm}
\end{addmargin}

It's the phrase that greets engineers as they type their first lines of code. My terminal, however, seems quite plain lately; the programs I write have transcended the computer screen and are mingling with everyday objects which I augment. In an abstract sense, I hope my artifacts exist to engage in a meaningful dialog with the physical world. As motivation for my technical and creative qualifications, I will describe how a simple everyday object can transcend the boundaries of the digital and physical... Many engineers write about their early experiences with Lego building blocks when describing their introduction to the subject. \emph{Instead, I will write about my experiences building the blocks themselves.}\\

Conventional toy building block sets are an important part of a child's learning and development process, allowing children to use their imagination and creativity to build a wide variety of representative structures. Each built structure exists only as a physical entity, these constructions must be dismantled when the blocks are needed to building new structures. Worse, they do not interact with the growing variety of virtual platforms with which children engage during play.\\

Virtual platforms afford children a constantly evolving variety of content-based play experiences that can engage imaginations with stories and character development, puzzles and challenges, and synchronous and asynchronous social play with children in the same or in remote locations.\\

I created a set of blocks that can be used for construction in a conventional, physical way but can also directly interact with virtual platforms in real time by displaying a mirrored digital shadow of the block configuration.\\

To achieve this, I developed a prototypical cyber-physical system complete with computation, control, and communication.  The blocks utilize a customized version of \emph{1-Wire}, a low-cost, minimal communications protocol from the HVAC industry in order to communicate state information amongst peers and a local processor. This protocol and the algorithms I developed allows for a rich user experience that was previously not mechanically or economically feasible. The result is the digital distillation of a child's imagination, where other people and smart objects -- regardless of whether or not they are in the same room -- can interact.\\

While designing unique interactions for others is an interesting technical and creative challenge, I want to study the design and control of agents conduct their own interactions with their world. Broadly, my research interests lie in the field of cyber-physical systems. Such systems perform complex tasks that cross the boundaries between number crunching and physical state evolution in the real world. If we are to trust these autonomous agents to make decisions about critical (and not so critical) tasks, then the next generation of vehicles, medical devices, sensors, and robotics must exist as or with proofs onto their own efficacy. In short, such systems must be engineered, not crafted.\\

As an undergraduate at the University of Pennsylvania I have already begun preparing myself for research in the field of cyber-physical systems via my coursework and extracurricular activities. My coursework can be roughly divided into two categories: theoretical underpinnings and hardware and software foundations.\\

On the theoretical front I have taken graduate courses in \emph{linear systems} and \emph{theory of embedded computation}, which cement the groundwork of continuous time state evolution and actuation. Furthermore, courses in \emph{probability} and \emph{stochastic processes} provide a foundation for many popular perception and planning algorithms, \emph{game theory} which provides a means of synthesizing multi-agent controllers, and finally  \emph{control theory} which bridges the gap between the practical and theoretical.\\

I have also developed practical skills to design and build both the software and hardware components of an embedded system through significant projects in courses including \emph{embedded systems}, \emph{operating systems}, and \emph{solid-state circuits}. Through my coursework,  as a Ph.D. student, I hope to continually add to my skill set, particularly by learning new mathematical and practical tools with which I may attack problems.\\

Furthermore, I believe the best way to develop understanding is to apply knowledge. Towards this end, I have supplemented my classroom education with rigorous independent projects that featured both theoretical and practical components. Last summer, as a an intern researcher at NASA Langley's Safety-Critical Avionics branch, I developed embedded software to optimize autonomous UAV flight plans. Successfully tackling this problem required both practical software development skills as well as an understanding of Bayesian algorithms. My results are to be published in [ref to paper], a white paper published by the branch.\\

In addition to the two projects I have mentioned thus far, I am currently working on my senior design project -- a multi-agent UAV system to help response teams conduct searches. Recent results from multi-agent planning will be incorporated to improve efficiency and robustness of the system. Research finds that safe, locally optimal search controllers can be generated from mission goals expressed as temporal logic [ref to paper]. Users only need to specify certain abstract goals, and the algorithm will determine an efficient way to achieve them. In short, this project summarizes my education thus far: by understanding \emph{theoretic abstractions}, we can design better \emph{physical systems}.\\

As a doctoral candidate, I intend to explore problems related to the verification and control of cyber-physical systems. As these systems continue to evolve and subsume tasks traditionally performed by humans, both their complexity and the harm a fault can cause will grow. Traditional engineering techniques are not capable of scaling to meet these increasing demands. Recently, researches have shown the efficacy of formal methods and controller synthesis to manage the complexity of such systems and prove their efficacy. I am most interested in the further development of these techniques and their relation to modern control theory. As such, I am most interested in working in the Cyber-Physical Systems group under Dr. Paulo Tabuada.\\

In summary, my undergraduate education in the theoretical and practical components of cyber-physical systems has prepared me to conduct research in planning and control. The extracurricular projects I have completed display my significant technical skills. The hardest problems in robotics are often ones that humans meet with little effort: moving, planning, coordinating, etc. I see the relationship between high-level planning and low-level control as a space ripe with important and fascinating problems. I hope that I will be at UCLA, developing unique solutions to these problems and enabling a more intelligent, safe, and efficient world.

\pagebreak

Moreover, I continually augment my formal education with practical experience and projects. I have always believed that one of the best ways to cement understanding is through application. As a high school student, I developed my own emulation of the GameBoy hardware from its specification so that I could better my understanding of computer architecture. Throughout my undergraduate experience, I have continued to work on extracurricular projects that pushed me to develop new skills. Last summer, I worked as a research intern at NASA Langley’s Safety-Critical Avionics branch, where I worked developing a Bayesian algorithm to optimize autonomous UAV missions subject to different fuel constraints. My work from my two-month period at NASA Langley is to appear as a whitepaper, entitled “Mission Optimization for Unmanned Aerial Vehicles,” that will be made available through the NASA Scientific and Technical Information Program. Additionally, I am currently working as an embedded systems engineer in a research group at Penn on a new serial communications protocol for low-cost and low-power distributed systems, with applications in many areas including modular robotics. Furthermore, in my undergraduate project courses, I have continually selected projects that feature a synthesis of different fields so I could apply what I have learned. As part of my Embedded Systems course, I developed a hardware-in-the-loop testing platform for hybrid electric vehicle power systems. For my capstone design project, I am working on applying recent research, such as [1] and [2] to develop a multi-agent UAV system for first responders to use after a natural disaster. I believe that the experience I have gained from these projects has prepared me for the projects that I will work on as part of my graduate research.

As a doctoral candidate, I wish to conduct research on the verification and control of multi-agent systems. Due to Moore’s Law, powerful computer systems are now both affordable and portable enough to include in any cyber-physical system. In the case of robotics, these computational platforms have allowed for significant advancement in multi-agent systems, since advances in the intelligence of each agent is seen many times over in the gestalt. I believe that, despite the difficulties present in controlling them, these systems show great potential due to their versatility. Their nature presents an effective solution to problems ranging from those require agents distributed across large distances, such as those in agriculture, to the precise actuation of a single object as in [3]. However, the interactions between the agents makes these systems more difficult to create and analyze, and any design failures can produce substantial damage. Recent research, such as [2], suggests that a combination of formal methods for high-level logic and modern low-level control techniques are necessary to properly orchestrate these multi-agent systems. It is the relationship between these two techniques that I wish to explore as a Ph.D. candidate. Due to my research interests, I am most interested in working with Dr. Mac Schwager or ANOTHER DUDE. Humanity is accelerating toward the visons of Gibson and Asimov, and I believe that multi-agent robotics will play an important role in this transition.

In conclusion, I am confident that my coursework as an undergraduate, as well as my past project work has prepared me to successfully transition into a doctoral candidate. I believe that multi-agent robotic systems show great promise as solutions to a wide range of problems, and I hope to push these systems toward greater viability at MIT.

\end{document}


