\documentclass[]{letter}
\usepackage{graphicx}
\usepackage{wrapfig}
\usepackage{lipsum}
\usepackage{scrextend}
\usepackage{verbatim, hyperref}
\usepackage[tmargin=1in,bmargin=1in,lmargin=1.25in,rmargin=1.25in]{geometry}
%opening
\title{}
\author{}

\begin{document}
\newcommand{\comm}[1]{}

Vincent Pacelli \hfill \hfill \href{mailto:pacelliv@seas.upenn.edu}{pacelliv@seas.upenn.edu}\\
\rule{\textwidth}{0.1pt}

As an undergraduate at the University of Pennsylvania, I have successfully prepared myself to begin conducting independent research in the field of cyber-physical systems as a Ph.D. candidate. Repeated rigorous coursework and independent projects provide me with the tools to attack difficult problems through both theoretical abstraction and practical implementation. These experiences uniquely qualify me to pursue graduate work in autonomous systems.


My coursework can be roughly divided into two categories: theoretical underpinnings and technical foundations.  On the theoretical front, I have completed or am now taking graduate courses in \emph{linear systems} and \emph{theory of embedded computation}, which provide the groundwork of continuous and discrete state evolution and actuation; \emph{probability} and \emph{stochastic processes}, which provide a foundation for many popular perception and planning algorithms; \emph{game theory}, which provides a means of synthesizing multi-agent controllers; and \emph{control theory}, which bridges the gap between the practical and theoretical aspects of developing such systems.  On the technical side, through significant projects in courses including \emph{embedded systems}, \emph{operating systems}, and \emph{solid-state circuits}, I have developed the practical skills necessary to design and build both the software and hardware components of an embedded system.  As a Ph.D. student, I hope to continue to add to my skill set, particularly by learning new mathematical and practical tools with which I may attack problems.

Because I believe that the best way to develop understanding is to apply knowledge gained from studies, I have supplemented my classroom education with rigorous independent projects that featured both theoretical and practical components. Prior to my arrival at Penn, I developed a software emulation of the Nintendo Gameboy in an effort to better understand computer architecture. Then, in my second year, I gained practical experience developing research software for characterizing infant jaw motion, some of which is still actively being used by hospitals. More recently, I interned at NASA Langley's Safety Critical Avionics Branch, where I helped develop software for routing UAVs in the presence of changing mission parameters. My work will be published as part of an upcoming branch white paper. In addition, I am actively involved as an embedded systems engineer in Penn’s xLab, where I help produce resource-constrained embedded systems for context-aware toys.  Ensuring a robust user experience entails meeting strict performance and size requirements and which requires creativity in design and skill in execution. Finally, in collaboration with Penn's GRASP Lab, I am currently working on my senior design project -- a multi-agent UAV system to help disaster response teams conduct searches. We intend to incorporate recent results from multi-agent planning research, such as mission synthesis from logic specifications, to improve the efficiency, robustness, and usability of the system. In essence, this project epitomizes the major theme of my engineering education to date: by understanding \emph{theoretic abstractions}, we can design better \emph{physical systems}.

The rapid proliferation of cyber-physical systems has underscored the importance of this motif: as we move towards a more connected, autonomous world, how can we trust these systems to behave both efficiently and safely? If we are to be able to trust autonomous agents to make decisions about critical (and not so critical) tasks, then the next generation of vehicles, medical devices, sensors, and robotics must be accompanied by \emph{proofs of their own efficacy}. In short, such systems must be engineered, not crafted, and this is where my research interests lies.

More specifically, I intend to explore problems that relate the planning and control of autonomous systems.  While Moore's Law provides us with increasingly smaller, more efficient computing platforms, brute force alone cannot achieve the sophisticated interactions promised by futurists like Asimov and Roddenberry and which are worthy of our aspiration.  Instead, we need to continue to develop intelligent algorithms to solve problems that humans do naturally, such as where to go and how to get there.  Recent techniques, such as “correct-by-construction” controller synthesis and trajectory generation via LQR-Trees, have greatly improved the tractability of autonomous systems but significant further research is still required -- especially in scenarios involving dangerous obstacles and non-linear or stochastic systems.  Due to the intricacy of these problems and the importance of finding solutions to them in our increasingly augmented world, I relish the opportunity to address them, and, in this regard, I would be very interested in working with Dr. Sertac Karaman, Dr. Russ Tedrake, Dr. Brian Williams, or Dr. Nicholas Roy.

In summary, my undergraduate education and work experience in the theoretical and practical components of cyber‑physical systems has prepared me to conduct research in autonomous systems. The extracurricular projects I have completed display my significant technical skills. In my experience, the most difficult problems in cyber- physical systems are often ones that humans meet with little effort: moving, planning, coordinating, etc. I see the relationship between high-level planning and low-level control as a space replete with important and fascinating problems. I hope to have the opportunity to be part of the MIT team that is developing unique solutions to these problems and enabling a more intelligent, safe, and efficient world.
%\rule{\textwidth}{1pt}


\end{document}